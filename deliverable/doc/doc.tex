\documentclass[portuguese, 11pt, a4paper,titlepage, oneside]{article}

\usepackage[margin=2.5cm]{geometry}
\usepackage[T1]{fontenc}
\usepackage{graphicx}		
\usepackage{amssymb}
\usepackage[portuguese]{babel}
\usepackage[framed, numbered]{matlab-prettifier}
\usepackage{subcaption}
\usepackage{wrapfig}
\usepackage{hyperref}

% Title page config
\title{
  \includegraphics[width=6cm]{assets/logo-deti-black.pdf}
  \vskip 2.5em
  \Huge Métodos Probabilísticos para\\
  Engenharia Informática
  \vskip .7em
	{\bfseries Guião PL04}
  \vskip 1.5em
  \normalsize Ano letivo 2023/2024
  \vskip 0.5em
  Turma P4
}

\author{Ricardo Quintaneiro\\NMec: 110056\and Rúben Garrido\\NMec: 107927}
\date{\today}

% Table of contents config
\addto\captionsportuguese{\renewcommand*\contentsname{Índice}}

% Begin document
\begin{document}

% Title
\maketitle

% Table of contents
\tableofcontents
\pagebreak

% Content
\section{Introdução}
No âmbito da Unidade Curricular de Métodos Probabilísticos para Engenharia Informática, foi-nos proposto realizar um trabalho prático relativo a Hash Functions, Bloom Filter e MinHash. Este consiste em desenvolver um script de consulta de filmes e géneros, que se relacionam entre si.

Este relatório serve para documentar os processos de programação dos scripts e para fundamentar as escolhas das opções relativas à implementação dos métodos probabilísticos, como por exemplo, o número de funções de dispersão a usar, o tamanho de shingles e a dimensão dos filtros de Bloom.

\section{Material fornecido}
Foi-nos fornecido um ficheiro CSV (\textit{Comma-Separated Values}), \verb|movies.csv|, onde constam informações sobre mais de 50000 filmes.

Quanto à estrutura, na primeira coluna está presente o título, na segunda o ano e nas restantes os vários géneros associados ao filme.

\pagebreak
\section{Desenvolvimento}
\subsection{Hash function}
Recorremos à função \verb|muxDJB31MA| para efetuar o \textit{hashing} em todas as ocasiões onde este é necessário.

Esta é uma modificação da função \verb|DJB31MA| utilizada nas aulas, onde é usado um argumento \verb|k| correspondente ao número de funções de \textit{hash}. Isto permite reduzir a complexidade computacional, já que deixa de ser necessário invocar a função e percorrer a string para cada função de \textit{hash}.

\begin{lstlisting}[style=Matlab-editor]
function aux = muxDJB31MA(chave, seed, k)
len = length(chave);
chave = double(chave);
h = seed;
aux = zeros(1, k);
for i=1:len
    h = mod(31 * h + chave(i), 2^32 -1) ;
end
for i = 1:k
    h = mod(31 * h + i, 2^32 -1) ;
    aux(i) = h;
end
end
\end{lstlisting}

\subsection{Opção 1}
Filtrámos, através de um ciclo \verb|for|, os demais géneros, de modo a obter um vetor de géneros únicos, ignorando os elementos do \textit{cell array} que estão \textit{missing}.

\begin{lstlisting}[style=Matlab-editor]
genres = strings([]);
[x, y] = size(movies);

for i = 1:x
    for j = 3:y
        if ~ismissing(movies{i, j})
            g = movies{i,j};
            if ~strcmp(g, '(no genres listed)') && ~ismember(convertCharsToStrings(movies{i, j}), genres)
                genres = [genres convertCharsToStrings(movies{i, j})];
            end
        end
    end
end
\end{lstlisting}

\subsection{Opção 2}
\subsubsection{Criação do filtro de Bloom}
Para inicializar um filtro de bloom, recorre-se a um vetor com o tamanho passo como argumento da função \verb|bloomFilterInitialization|.

Recorremos à formula \(\frac{-n\times log(n)}{log(2)^2}\) para obter o tamanho adequado do filtro, onde \(n\) é a probabilidade de falsos positivos desejada. Escolhemos \(n\) como \(0.001\) porque consideramos ser um equilibro entre uma percentagem baixa de erro e o consumo de memória do filtro.

\begin{lstlisting}[style=Matlab-editor]
m = ceil(-n*log(0.001)/(log(2))^2);
genre_bloom = bloomFilterInitialization(m);

function bloom = bloomFilterInitialization(n)
bloom = zeros(1, n, 'uint16');
end
\end{lstlisting}

\subsubsection{Inserção de elementos} \label{2_insert}
Para inserir elementos, utilizamos a função \verb|bloomFilterInsert|, onde se obtém o conjunto de \textit{hashes} com base na \textit{key} inserida, e, ao valor, se calcula o resto da divisão pelo tamanho do filtro de Bloom. Uma vez que este filtro envolve contagem, é feita a soma de 1 ao invés de definir como \textit{true}.

Utilizámos esta função para inserir todos os géneros que se encontram no cell array criado pelo ficheiro CSV, ignorando os valores \textit{missing}.

\begin{lstlisting}[style=Matlab-editor]
n = length(genres);
m = ceil(-n*log(0.001)/(log(2))^2);
k_g = round((m/n)*log(2));

for i = 1:x
    for j = 3:y
        if ~ismissing(movies{i, j})
            genre_bloom = bloomFilterInsert(genre_bloom, movies{i, j}, k_g);
        end
    end
end

function bloom = bloomFilterInsert(bloom, key, k)
m = length(bloom);
aux = muxDJB31MA(key, 127, k);
for i = 1:k
    hash = mod(aux(i), m) + 1;
    bloom(hash) = bloom(hash) + 1;
end
end
\end{lstlisting}

\subsubsection{Pesquisa}
Ao \textit{char array} obtido através do \textit{input}, é feita uma conversão para \textit{string}, onde depois se invoca a função \verb|bloomFilterCheck|.

Esta possui a mesma lógica da função \verb|bloomFilterInsert| (ver secção \ref{2_insert}), embora guarde os vários valores obtidos pela função de \textit{hash} e retorne o menor.

Caso o valor obtido seja 0, significa que o género não existe.

\begin{lstlisting}[style=Matlab-editor]
genre = convertStringsToChars(input("Select a genre: ", "s"));
if ~ismember(genre, genres)
    fprintf("Genre not found!\n")
else
    count = bloomFilterCheck(genre_bloom, genre, k);
    fprintf("%d movies with genre %s\n", count, genre);
end

function count = bloomFilterCheck(bloom, key, k)
    m = length(bloom);
    aux = muxDJB31MA(key, 127, k);
    count = [];
    for i = 1:k
        key = [key num2str(i)];
        hash = mod(aux(i), m) + 1;
        count = [count bloom(hash)];
    end
    
    if ~isempty(count)
        count = min(count);
    else
        count = 0;
    end
end
\end{lstlisting}

O output desta opção é o seguinte:
\begin{lstlisting}[style=Matlab-editor]
Select a genre: 
  Comedy

16124 movies with genre Comedy
\end{lstlisting}


\subsection{Opção 4}
\subsubsection{Estrutura de dados} \label{4_structure}
Para criar as assinaturas necessárias à execução desta opção, é necessário criar uma estrutura de dados adequada, que permita posteriormente efetuar a comparação entre a string introduzida e os vários títulos.

Assim, para cada filme, são criados shingles a partir do título. Estes têm comprimento 2, uma vez que é o número que permite obter uma maior fiabilidade nos resultados, pelo facto de existir uma maior granularidade.

\begin{lstlisting}[style=Matlab-editor]
[Set_title, Nm] = createMovieTitleStructure(movies);

function [Set, Nm] = createMovieTitleStructure(movies)
% For each movie, get its title split in shingles.
Nm = length(movies);
Set = cell(Nm,1);

for n = 1:Nm % Lines (movies)
    title = movies{n,1};
    for i = 1:length(title)-1
        Set{n} = [Set{n} convertCharsToStrings(title(i:i+1))];
    end
end
end
\end{lstlisting}

\subsubsection{Assinaturas} \label{4_signatures}
As assinaturas são geradas com a função \verb|getSignatures|. Para cada conjunto do set, esta função faz uso do algoritmo MinHash: calcula a \textit{hash} para cada elemento e guarda o menor valor obtido.

Foram utilizadas 100 funções de hash (ou seja, \(K = 100\)), uma vez que consideramos ser um número adequado tendo em conta o tamanho do set em questão.
\begin{lstlisting}[style=Matlab-editor]
signatures_title = getSignatures(Set_title, Nm, 100);

function signatures = getSignatures(Set, Nm, K)
signatures = inf(Nm, K);

for n = 1:Nm
    set_n = Set{n};
    for i = 1:length(set_n)
        key = num2str(set_n(i));
        h_out = muxDJB31MA(key, 127, K);
        signatures(n,:) = min(h_out, signatures(n,:));
    end
end
end
\end{lstlisting}

\subsubsection{Pesquisa}
Para a string introduzida na consola do MATLAB, são repetidos os passos das secções \ref{4_structure} e \ref{4_signatures}, de modo a obter as suas assinaturas.

\begin{lstlisting}[style=Matlab-editor]
a = input("Insert a string: ", "s");
Set = cell(1,1);

for i = 1:length(a)-1
    Set{1} = [Set{1} convertCharsToStrings(a(i:i+1))];
end

k = size(signatures_title, 2);
signatures = getSignatures(Set, 1, k);
\end{lstlisting}

De seguida, comparam-se ambas as matrizes de assinaturas para obter a distância (e consequentemente a similaridade) de Jaccard.

\begin{lstlisting}[style=Matlab-editor]
size_titles = size(signatures_title, 1);
similarities = zeros(1, size_titles);


for n = 1:size_titles
    similarities(n) = 1 - (sum(signatures ~= signatures_title(n, :)) / k);
end
\end{lstlisting}

Após obter o vetor de similaridades, é efetuada neste uma ordenação decrescente quanto à similaridade de Jaccard e aplicado um limite de 5 filmes, de acordo como é pedido no enunciado.
\begin{lstlisting}[style=Matlab-editor]
[sortedSimilarities, indices] = sort(similarities, 'descend');
topSimilarities = sortedSimilarities(1:5);
topTitles = movies(indices(1:5));
topGenres = Set_genre(indices(1:5));
\end{lstlisting}

Por fim, e uma vez que temos um conjunto de shingles, efetuamos uma concatenação entre estes de modo a obter o título completo.

\begin{lstlisting}[style=Matlab-editor]
fprintf("Top 5 Similar Titles:\n");
concatenatedTitle = strings(length(topTitles{i}) + 1);
for i = 1:5
    displayString = sprintf("\t%s - %f - Genres: ", topTitles{i}, topSimilarities(i));
    for j = 1:length(topGenres{i})
        displayString = strcat(displayString, convertCharsToStrings(topGenres{i}{j}), ", ");
    end

    fprintf('%s\n', displayString);
end
\end{lstlisting}

O output é, conforme esperado:
\begin{lstlisting}[style=Matlab-editor]
Insert a string: 
  Love

Top 5 Similar Titles:
  Love - 1.000000
  Lover - 0.690000
  Love65 - 0.530000
  Loveling - 0.520000
  Lovelife - 0.520000
\end{lstlisting}

\subsection{Opção 5}
\subsubsection{Estrutura de dados}
A estrutura de dados é semelhante à encontrada para a opção 4 (ver secção \ref{4_signatures}). No entanto, ao invés de criar \textit{shingles}, o set inclui os conjuntos de géneros para cada filme.

\begin{lstlisting}[style=Matlab-editor]
[Set_genre, Nm] = createMovieGenreStructure(movies);

function [Set, Nm] = createMovieGenreStructure(movies)
% For each movie, get its genres
Nm = length(movies);
Set = cell(Nm,1);

for n = 1:Nm % Lines (movies)
    for g = 3:12 % Columns (genres)
        if ~ismissing(movies{n, g})
            Set{n} = [Set{n} convertCharsToStrings(movies{n,g})];
        end
    end
end
end
\end{lstlisting}

\subsubsection{Assinaturas}
O conjunto de assinaturas para esta opção foi obtido da mesma forma que a opção 4, utilizando a \textit{hash function} (ver secção \ref{4_signatures}).

\begin{lstlisting}[style=Matlab-editor]
signatures_genre = getSignatures(Set_genre, Nm, 100);
\end{lstlisting}

\subsubsection{Pesquisa}
É pedida uma string de géneros no formato CSV pelo input do MATLAB, ao que é efetuado um split por vírgula, de modo a obter os diversos géneros que nesta se encontram.

\begin{lstlisting}[style=Matlab-editor]
a = input("Select one or more genres (separated by ','): ", "s");
a = strsplit(a, ',');
\end{lstlisting}

De seguida, é criada uma matriz de assinaturas para os vários géneros, tal como na opção 4.
\begin{lstlisting}[style=Matlab-editor]
Set = strings(length(a),1);
for i= 1:length(a)
    Set(i,1) = convertCharsToStrings(a(i));
end
b = cell(1,1);
b{1} = Set;
k = size(signatures_genre, 2);
signatures = getSignatures(b, length(b), k);
\end{lstlisting}

É efetuada uma comparação de assinaturas para obter a similaridade de Jaccard, seguido de uma ordenação e limite de output.
\begin{lstlisting}[style=Matlab-editor]
size_genre = size(signatures_genre, 1);
similarities = zeros(1, size_genre);

for n = 1:size_genre
    similarities(n) = 1 - (sum(signatures ~= signatures_genre(n, :)) / k);
end

[sortedSimilarities, indices] = sort(similarities, 'descend');
topSimilarities = sortedSimilarities(1:5);
topTitles = movies(indices(1:5), 1);
\end{lstlisting}

Por fim, é realizado um print dos vetores, que contêm os demais títulos e similaridades.
\begin{lstlisting}[style=Matlab-editor]
fprintf("Top 5 Similar Titles:\n");
for i = 1:5
    fprintf("\t%s - %f\n", topTitles{i}, topSimilarities(i));
end
\end{lstlisting}

O output é, como esperado:
\begin{lstlisting}[style=Matlab-editor]
Select one or more genres (separated by ','): 
  Comedy,Action,IMAX

Top 5 Similar Titles:
  Night at the Museum: Battle of the Smithsonian - 1.000000
  Men in Black III (M.III.B.) (M.I.B.) - 0.770000
  I Love Trouble - 0.750000
  Naked Gun 33 1/3: The Final Insult - 0.750000
  Low Down Dirty Shame, A - 0.750000
\end{lstlisting}

\end{document}