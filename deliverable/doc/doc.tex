\documentclass[portuguese, 11pt, a4paper,titlepage, oneside]{article}

\usepackage[margin=2.5cm]{geometry}
\usepackage[T1]{fontenc}
\usepackage{graphicx}		
\usepackage{amssymb}
\usepackage[portuguese]{babel}
\usepackage[framed, numbered]{matlab-prettifier}
\usepackage{subcaption}
\usepackage{wrapfig}
\usepackage{hyperref}

% Title page config
\title{
  \includegraphics[width=6cm]{assets/logo-deti-black.pdf}
  \vskip 2.5em
  \Huge Métodos Probabilísticos para\\
  Engenharia Informática
  \vskip .7em
	{\bfseries Guião PL04}
  \vskip 1.5em
  \normalsize Ano letivo 2023/2024
  \vskip 0.5em
  Turma P4
}

\author{Ricardo Quintaneiro\\NMec: 110056\and Rúben Garrido\\NMec: 107927}
\date{\today}

% Table of contents config
\addto\captionsportuguese{\renewcommand*\contentsname{Índice}}

% Begin document
\begin{document}

% Title
\maketitle

% Table of contents
\tableofcontents
\pagebreak

% Content
\section{Introdução}
No âmbito da Unidade Curricular de Métodos Probabilísticos para Engenharia Informática, foi-nos proposto realizar um trabalho prático relativo a Hash Functions, Bloom Filter e MinHash. Este consiste em desenvolver um script de consulta de filmes e géneros, que se relacionam entre si.

Continuar

\section{Material fornecido}
Foi-nos fornecido um ficheiro CSV (\textit{Comma-Separated Values}), \verb|movies.csv|, onde constam informações sobre mais de 50000 filmes.

Quanto à estrutura, na primeira coluna está presente o título, na segunda o ano e nas restantes os vários géneros associados ao filme.

\pagebreak
\section{Desenvolvimento}
\subsection{Hash function}
Recorremos à função \verb|muxDJB31MA| para efetuar o \textit{hashing} em todas as ocasiões onde este é necessário.

Esta é uma modificação da função \verb|DJB31MA| utilizada nas aulas, onde é usado um argumento \verb|k| correspondente ao número de funções de \textit{hash}. Isto permite reduzir a complexidade computacional, já que deixa de ser necessário invocar a função e percorrer a string para cada função de \textit{hash}.

\begin{lstlisting}[style=Matlab-editor]
function aux = muxDJB31MA(chave, seed, k)
len = length(chave);
chave = double(chave);
h = seed;
aux = zeros(1, k);
for i=1:len
    h = mod(31 * h + chave(i), 2^32 -1) ;
end
for i = 1:k
    h = mod(31 * h + i, 2^32 -1) ;
    aux(i) = h;
end
end
\end{lstlisting}

\subsection{Opção 1}
Filtrámos, através de um ciclo \verb|for|, os demais géneros, de modo a obter um vetor de géneros únicos, ignorando os elementos do \textit{cell array} que estão \textit{missing}.

\begin{lstlisting}[style=Matlab-editor]
genres = {};
[x, y] = size(movies);

for i = 1:x
    for j = 3:y
        if ~ismissing(movies{i, j})
            if ~ismember(convertCharsToStrings(movies{i, j}), genres)
                genres = [genres convertCharsToStrings(movies{i, j})];
            end
        end
    end
end
\end{lstlisting}

\begin{lstlisting}[style=Matlab-editor]
%% Sample Matlab code
!mv test.txt test2.txt
A = [1, 2, 3;... foo
     4, 5, 6];
s = 'abcd';
for k = 1:4
  Disp(s(k)) % bar
end
%{
create row vector x, then reverse it
%}
x = linspace(0,1,101);
y = x(end:-1:1);
\end{lstlisting}

\end{document}